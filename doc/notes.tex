\documentclass[11pt, oneside]{article}   	% use "amsart" instead of "article" for AMSLaTeX format
\usepackage{geometry}                		% See geometry.pdf to learn the layout options. There are lots.
\geometry{letterpaper}                   		% ... or a4paper or a5paper or ... 
%\geometry{landscape}                		% Activate for rotated page geometry
%\usepackage[parfill]{parskip}    		% Activate to begin paragraphs with an empty line rather than an indent
\usepackage{graphicx}				% Use pdf, png, jpg, or eps§ with pdflatex; use eps in DVI mode
								% TeX will automatically convert eps --> pdf in pdflatex		
\usepackage{amssymb}
\usepackage{mathtools}

%SetFonts

%SetFonts


\title{Brief Article}
\author{The Author}
%\date{}							% Activate to display a given date or no date

\begin{document}
\maketitle
%\section{}
%\subsection{}
The SN rate for a galaxy at a cosmic time $t_0$ is
\begin{align}
R_G &= \int_{0}^{\infty} \psi(\tau) \phi(t_0-\tau) d\tau \\
 &= \int_{-\infty}^{t_0} \psi(t_0-\tau) \phi(\tau) d\tau,
\end{align}
$\psi$ is the SFH, $\phi$ is the DTD. 
The SFH is a function of cosmic time (since the big bang).
The DTD has the form
\begin{equation}
\phi(\tau) =  
\begin{cases*} 
            0  &  if $\tau < t_p$  \\
             A \left(\frac{\tau}{\mathrm{Gyr}} \right)^\beta  & if $\tau \geq t_p$ 
          \end{cases*}.
\end{equation}
Then a useful expression for efficient calculation of the integral is
\begin{align}
R_G &= \frac{1}{1+\beta}  \int_{-\infty}^{t_0} \psi(t_0-\tau)  \phi(\tau)\tau^{-\beta}  d(\tau^{1+\beta}). \\
 &= \frac{A}{1+\beta}  \left(\frac{1}{\mathrm{Gyr}} \right)^\beta \int_{t_p}^{t_0} \psi(t_0-\tau) d(\tau^{1+\beta}) \\
 & = A   \left(\frac{1}{\mathrm{Gyr}} \right)^\beta  \int_{0}^{t_0-t_p} \psi(\tau) (t_0-\tau)^{\beta} d\tau .
 \end{align}
\end{document}  
